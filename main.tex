\documentclass[a4paper,10pt]{scrartcl}
\usepackage[ngerman]{babel}
\usepackage[T1]{fontenc}
\usepackage[utf8]{inputenc}
\usepackage{float}
\title{BlackWood}
\author{Benjamin Altmiks, Lukas Kleybolte, Dejan Pekez, Patrick Kostas}
\date{\today}
\begin{document}
\maketitle
\section{Aufgabe}
Wir von BlackWood bieten eine breite Produktpalette von Getränken, Lebensmitteln und Drogerieartikeln an.\newline
Alle unsere Produkte besitzen ein Gewicht, ein Mindesthaltbarkeitsdatum, Name, Lagerplatz, Barcode und eine Produktkategorie.
Unter den Getränken bieten wir alkoholische und nichtalkoholische Getränke an.\newline
Unsere Lebensmittel decken eine breite Bandbreite von Obst, Gemüse, Fleisch, Milchprodukte, Süßigkeiten, Tiefkühlartikeln und Backwaren an.\newline
Regionale Landwirte sind Zuliefer für Obst, Gemüse, Fleisch und Milchprodukte.\newline
Unsere Getränke beziehen wir alle von einer Stadtbrauerei.\newline
Die Wolf-Bäckerei beliefert uns mit Backwaren.\newline
In unserem Betrieb arbeiten Kassierer, Verkäufer, Lagerarbeiter, ein Marktleiter und ein stellvertretender Marktleiter.\newline
Kassierer und Verkäufer arbeiten im Verkaufsraum, Lagerarbeiter im Lagerraum und die Marktleiter arbeiten zusammen im Büro.\newline
Alle Mitarbeiter besitzen einen Namen, eine Personalnummer und einen Geburtstag.\newline
Unsere Kunden unterscheiden sich zwischen Privatkunden und Geschäftskunden.\newline
Ein Privatkunde besitzt eine Privatkundenkarte. Geschäftskunden besitzen eine Geschäftskundenkarte, wodurch sie von den  Mehrwertsteuern befreit sind.\newline
\newpage


\section{Aufgabe}
Blackwood bietet Getränke an.\newline
Blackwood bietet Lebensmitel an.\newline
Blackwood bietet Drogerieartikeln an.\newline
\newline
Produkte besitzen ein Gewicht.\newline
Produkte besitzen ein Mindesthaltbarkeitsdatum.\newline
Produkte besitzen einen Namen.\newline
Produkte besitzen ein Lagerplatz.\newline
Produkte besitzen einen Barcode.\newline
Produkte besitzen eine Produktkategorie.\newline
\newline
Produkte identifizieren sich durch einen Barcode \newline
Produkte identifizieren sich durch einen Namen \newline
Produkte identifizieren sich durch einen Gewicht \newline
\newline
Getränke werden von Stadtbrauereien geliefert.\newline
Getränke werden angeboten als alkoholische Getränke.\newline
Getränke werden angeboten als nichtalkoholische Getränke.\newline
\newline
Als Lebensmittel bieten wir Obst an.\newline
Als Lebensmittel bieten wir Gemüse an.\newline
Als Lebensmittel bieten wir Fleisch an.\newline
Als Lebensmittel bieten wir Milchprodukte an.\newline
Als Lebensmittel bieten wir Süßigkeiten an.\newline
Als Lebensmittel bieten wir Tiefkühlartikeln an.\newline
Als Lebensmittel bieten wir Backwaren an.\newline
\newline
Obst liefern regionale Landwirte.\newline
Gemüse liefern regionale Landwirte.\newline
Fleisch liefern regionale Landwirte.\newline
Milchprodukte liefern regionale Landwirte.\newline
\newline
Die Wolf-Bäckerei liefert Backwaren.\newline
\newline
In unserem Betrieb arbeiten Kassierer.\newline
In unserem Betrieb arbeiten Verkäufer.\newline
In unserem Betrieb arbeiten Lagerarbeiter.\newline
In unserem Betrieb arbeitet ein Marktleiter.\newline
In unserem Betrieb arbeitet ein stellvertretender Marktleiter.\newline
\newline
Kassierer arbeiten im Verkaufsraum.\newline
Verkäufer arbeiten im Verkaufsraum.\newline
Lagerarbeiter arbeiten im Lagerraum.\newline
Ein Marktleiter arbeitet im Büro.\newline
Ein stellvertretender Marktleiter arbeitet im Büro.\newline
\newline
Mitarbeiter haben Namen.\newline
Mitarbeiter haben eine Personalnummer.\newline
Mitarbeiter haben einen Geburtstag.\newline
\newline
Kunden sind Geschäftskunden.\newline
Kunden sind Privatkunden.\newline
\newline
Ein Privatkunde besitzt eine Privatkundenkarte.\newline
Ein Geschäftskunde besitzt eine Geschäftskundenkarte.\newline
Eine Geschäftskundenkarte befreit von Mehrwertsteuern.\newline

\subsection{Aufgaben}
Produkte: Eigenschaften von Produkten (auch Lagernummer)
Personen: Vorname, Nachname, geburtsdatum mit or zu:
Kunden: Bestellnummer, Rechnungsnummer mit xor-Beziehung
Geschäftskunden: Firma
Privatkunden: 
Mitarbeiter: Persöhnliche Daten
Geschäft mit 
Lagerraum: Stellplatznummer, Lagernnummer
Verkaufsraum: Verschiedene Bereiche mit Artikeln
Lieferanten: Firmenname, Firmenadresse, Telefonnummer   


\end{document}
